\section{The Homology of a Group}
\subsection{The Definition of $H_\ast G$}
\label{chap2.1}
\begin{definition}
Let $G$ be a group and $\varepsilon: F \rightarrow \mathbb{Z}$ a projective resolution of $\mathbb{Z}$ over $\mathbb{Z} G$. We define the homology groups of $G$ by
\[
H_{i} G=H_{i}\left(F_{G}\right)
\]
where $M_{G}=M\otimes_{\mathbb{Z}G}\mathbb{Z}$ is the group of co-invariants of a $G$-module $M$.
\end{definition}
We need to examine that the right-hand side is independent of the choice of resolution, up to canonical isomorphism. 
\begin{theorem}
Given projective resolutions $F$ and $F^{\prime}$ of a module $M,$ there is an augmentation-preserving chain map $f: F \rightarrow F^{\prime},$ unique up to homotopy, and $f$
is a homotopy equivalence.
\end{theorem}
\begin{proof}
See \cite{brown2012cohomology}cf.I(7.5)
\end{proof}

Therefore we can compute the homology of a group via any resolution. 

\begin{example}
\label{example1}
suppose $G$ is a finite cyclic group of order $n$. Using the resolution
\[
\ldots \stackrel{t-1}{\longrightarrow} \mathbb{Z} G \stackrel{N}{\rightarrow} \mathbb{Z} G \stackrel{t-1}{\longrightarrow} \mathbb{Z} G \rightarrow \mathbb{Z} \rightarrow 0
\]
We obtain the complex
\[
\cdots \stackrel{0}{\rightarrow} \mathbb{Z} \stackrel{n}{\rightarrow} \mathbb{Z} \stackrel{0}{\rightarrow} \mathbb{Z}
\]
Thus
\[
H_{i} G \simeq \left\{\begin{array}{ll}
\mathbb{Z} & i=0 \\
\mathbb{Z}_{n} & i \text { odd } \\
0 & i \text { even }
\end{array}\right.
\]
\end{example}

\subsection{Topological Interpretation}
Via the complex of $K(G,1)$ space we can get a free resolution over $\mathbb{Z}G$. We will prove the following in this section:
\begin{theorem}
\label{thm1}
If $Y$ is a $K(G,1)$-complex then $H_\ast G\simeq H_\ast Y $
\end{theorem}
We begin with some definitions
\begin{definition}
By a $G$-complex we will mean a CW-complex X together with an action of G on X which permutes the cells.
\end{definition}
Thus we have for each $g \in G$ a homeomorphism $x \mapsto g x$ of $X$ such that the image $g \sigma$ of any cell $\sigma$ of $X$ is again a cell. For example, if $X$ is a simplicial complex on which $G$ acts simplicially, then $X$ is a $G$-complex.
\par
If $X$ is a $G$-complex then the action of $G$ on $X$ induces an action of $G$ on the cellular chain complex $C_{*}(X),$ which thereby becomes a chain complex of $G$ -modules. Moreover, the canonical augmentation $\varepsilon: C_{0}(X) \rightarrow \mathbb{Z}$ (defined by $\varepsilon(v)=1$ for every 0 -cell $v$ of $X$ ) is a map of $G$ -modules.
\par
\begin{definition}
$X$ is a free $G$-complex if the action of $G$ freely permutes the cells of $X$ (i.e., $g \sigma \neq \sigma$ for all $\sigma$ if $g \neq 1$ ). 
\end{definition}
In this case each chain module $C_{n}(X)$ has a $\mathbb{Z}$ -basis which is freely permuted by $G,$ hence $ C_{n}(X)$ is a free $\mathbb{Z} G$ -module with one basis element for every $G$-orbit of cells. 
\par
Finally, if $X$ is contractible, then $H_{*}(X) \approx H_{*}(\mathrm{pt} .) ;$ in other words, the sequence
\[
\cdots \rightarrow C_{n}(X) \stackrel{\partial}{\rightarrow} C_{n-1}(X) \rightarrow \cdots \rightarrow C_{0}(X) \stackrel{\varepsilon}{\rightarrow} \mathbb{Z} \rightarrow 0
\]
is exact. We have, therefore:
\begin{proposition}
\label{prop1}
Let $X$ be a contractible free $G$-complex. Then the augmented cellular chain complex of $X$ is a free resolution of $\mathbb{Z}$ over $\mathbb{Z} G$
\end{proposition}
If $Y$ is a $K(G, 1)$ then the universal cover $p: X \rightarrow Y$ is a regular cover(i.e. the deck transformations act transitively on the fibre) whose group of deck transformations is isomorphic to $\pi_{1} Y=G$.  We therefore obtain from proposition \ref{prop1}:
\begin{proposition}
If $Y$ is a $K(G, 1)$ then the augmented cellular chain complex of the universal cover of $Y$ is a free resolution of $\mathbb{Z}$ over $\mathbb{Z} G$
\end{proposition}
Now we move to the key lemma:
\begin{lemma}
Let $X$ be a free $G$-complex and let $Y$ be the orbit complex
$X / G .$ Then $C_{*}(Y) \approx C_{*}(X)_{G}$.
\end{lemma}
\begin{proof}
The projection $C_{*}(X) \rightarrow C_{*}(Y)$ induces by passage to the quotient $\operatorname{arap} \varphi: C_{*}(X)_{G} \rightarrow C_{*}(Y) .$ Now $C_{*}(X)_{G}$ has a $\mathbb{Z}$-basis with one basis element for each $G$-orbit of cells of $X$(According to the definition, If $F$ is a free $\mathbb{Z} G$ -module with basis $\left(e_{i}\right),$ then $F_{G}$ is a free $\mathbb{Z}$-module with basis $\left(\bar{e}_{i}\right)$). But $C_{*}(Y)$ also has a $\mathbb{Z}$ -basis with one element for each $G$-orbit of cells of $X,$ and it is clear that $\varphi$ maps a basis element of $C_{*}(X)_{G}$ to the corresponding basis element of $C_{*}(Y),$ hence $\varphi$ is an isomorphism.
\end{proof}
Let X be the universal cover of Y, the $K(G,1)$-complex, then we have $C_{*}(Y) \approx C_{*}(X)_{G}$ and thus proved the theorem \ref{thm1}.
