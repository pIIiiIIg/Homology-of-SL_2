\section{Introduction}
This is a note written for the final project of the course Group Cohomology. In this note I state how to calculate the homology of $SL_2(\mathbb{Z})$. 
\par
To begin with, I define what is the homology $H_\ast G$ of a group G in chapter \ref{chap2.1} and then show that this is equal to the homology of the K(G,1)-complex in theorem \ref{thm1}.  Using the Seifert-van Kampen theorem. I proved the Whitehead theorem \ref{thm2}, which says the functor $K(-, 1)$ preserves push-out, and so I get the Mayor-Vietoris sequence for homology of the amalgamation product $G=G_{1} *_{A} G_{2}$(corollary \ref{cor1}).
\par
Later, I show in chapter \ref{chap3} that the group action on a tree somehow determined the group. Especially, I proved theorem \ref{thm3.3} that when a group G acts without inversions on a tree in such a way that $G$ acts transitively on edges, then G can be presented as amalgamations. Luckily, the group action of $SL_2(\mathbb{Z})$ on the Farey Tree defined in chapter \ref{chap4.1} is proved to have such properties(chapter \ref{chap4.2}), and thus $S L_{2}(\mathbb{Z}) \cong \mathbb{Z} / 4 *_{\mathbb{Z} / 2} \mathbb{Z} / 6$(theorem \ref{thmain}) being shown. 
\par
Finally, by computing the Mayor-Vietoris sequence for homology of the amalgamation product of $S L_{2}(\mathbb{Z}) \cong \mathbb{Z} / 4 *_{\mathbb{Z} / 2} \mathbb{Z} / 6$, the homology of $S L_{2}(\mathbb{Z})$ is obtained(theorem \ref{thm3}).